\begin{abstract}
  \noindent
  Probabilities describe degrees of belief, and probabilistic inference describes rational reasoning under uncertainty. It is no wonder, then, that probabilistic models have exploded onto the scene of modern artificial intelligence, cognitive science, and applied statis- tics: these are all sciences of inference under uncertainty. But as probabilistic models have become more sophisticated, the tools to formally describe them and to perform probabilistic inference have wrestled with new complexity. Just as programming beyond the simplest algorithms requires tools for abstraction and composition, complex probabilistic modeling requires new progress in model representation—probabilistic programming languages. These languages provide compositional means for describing complex probability distributions; implementations of these languages provide generic inference engines: tools for performing efficient probabilistic inference over an arbitrary program ~\cite{goodman}.
  
  Probabilistic programs are usual functional or imperative programs with two added constructs: (1) the ability to draw values at random from distributions, and (2) the ability to condition values of variables in a program via observations. Models from diverse application areas such as computer vision, coding theory, cryptographic protocols, biology and reliability analysis can be written as probabilistic programs.
  
  Probabilistic programming aims to make the code of probabilistic models shorter, to reduce development time, to facilitate the construction of richer models, to require lower levels of expertise in building machine learning applications, and to support the construction of integrated models.
  
  Probabilistic inference is the problem of computing an explicit representation of the probability distribution implicitly specified by a probabilistic program. Depending on the application, the desired output from inference may vary—we may want to estimate the expected value of some function f with respect to the distribution, or the mode of the distribution, or simply a set of samples drawn from the distribution. In Probabilistic programming, modeling and inference have been disentangled.
  
  There are many existing probabilistic programming systems including BUGS ~\cite{bugs}, Church ~\cite{church}, FACTORIE ~\cite{factorie}, Infer.NET ~\cite{infernet}, Dimple ~\cite{dimple}, etc. The problem is that this is not easy for those cross-platform developments where they have to get accustomed to the different kinds of probabilistic programming languages or libraries. Henceforth, in this paper, what we proposed is the Portable Probabilistic Programming Framework that can be embedded in every programming language people commonly used. We designed the syntax for the portable probabilistic programming language which targets Bayesian networks and conditional query. The design of the language is based on BUGS ~\cite{bugs} and is more specific and efficient for describing the probabilistic models. The description of the models using the portable probabilistic programming language is separated from the code of the host language as well of the conditional query, which can enhance the reusability of the probabilistic models. The parser is implemented and the inference engine is generated automatically based on the conditional query. The inference algorithm is based on the MCMC sampling, such as Gibbs Sampling or Metropolis-Hastings Algorithm, which is efficient and lightweight to implement. Additionally, the APIs for other languages is attached leveraging some existing development tool such as SWIG(Simplified Wrapper and Interface Generator) ~\cite{swig}.
\end{abstract}

\keywords{
  Probabilistic programming language, probabilistic graphical model, probabilistic inference, embedded programming language
}
